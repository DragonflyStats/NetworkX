\subsection{Dixon's Q test}

In statistics, Dixon's Q test, or simply the Q test, is used for identification and rejection of outliers. 
This test should be used sparingly and never more than once in a data set. To apply a Q test for bad data, arrange the data in order of increasing values and calculate Q as defined:

\begin{equation}
Q = \frac{\mbox{Gap}}{\mbox{Range}}
\end{equation}

Where gap is the absolute difference between the outlier in question and the closest number to it. 
If $Q_calculated > Q_table$, then reject the questionable point.

%-------------------------------------------------------------------------------------------%
